\documentclass[12pt,a4paper]{article}
\usepackage[utf8]{inputenc}
\usepackage[left=2.5cm,right=2.5cm,top=3cm,bottom=2cm]{geometry}
\author{Pauline Speckmann}
\usepackage{graphicx}
\usepackage{booktabs}

\usepackage{fancyhdr}
\pagestyle{fancy}
\fancyhf{}
\fancyhead[l]{Digitalisierung Vorlesung 4 $-$ Zusammenfassung von Pauline Speckmann}
\fancyhead[r]{\thepage}

\begin{document}
\setcounter{section}{3}
\section{Konzept der digitalen Transformation}


\vspace*{1cm}
\subsection{Grundlagen digitale Transformation} %%%%%%%%%%%%%%%%%%%%%%%%%%%%%%%%%%%%%%%%%%%%%%%%%%%%%%%%%%%%%%%%%%%%%%%%%%%%%%%%%
\begin{itemize}
   \item \textbf{Potenziale von IT - Traditionelle Perspektive}
      \begin{itemize}
			\item Unstrukturierte Abläufe in routinemäßige Arbeit überführen
			\item Beschleunigung wertschöpfender Aktivitäten
			\item Ersatz und Reduktion menschlicher Arbeit
			\item Transport von Informationen mit großer Geschwindigkeit über große Entfernungen
			\item Große Menge von Informationen verfügbar machen
      \end{itemize}

   \item \textbf{Unternehmensstrategie und Informationssysteme}:
   \item[] \includegraphics[scale=0.45]{UIS.png}
   
   \item \textbf{Potenziale digitaler Technologien für die Wertschöpfung}:
   \item[] \includegraphics[scale=0.5]{Potenziale.png}
   
   \item \textbf{Digitale Transformation - Zentrale Schritte}:
      \begin{itemize}
			\item \textbf{Starke operationale Grundlage}: \\
			      Zuverlässige Kunden- und Produktdaten, End-to-End Transaktionsprozesse, Transparenz bei Kundentransaktionen
			\item \textbf{Experimentierfreudigkeit}: \\
			      Umfassende Einbindung von Mitarbeiter in Innovationsbemühungen
			\item \textbf{Datengesteuerte Entscheidungskultur}: \\
			      Hypothesenbildung, Datensammlung und detaillierte Auswertung, Top-Level Entscheidungskultur
			\item \textbf{Digitale Angebotsplattform}: \\
			      Wiederverwendbare Datentools und Algorithmen, Unterstützung bei der Konfiguration digitaler Lösungen
      \end{itemize}
\end{itemize}


\vspace*{0.5cm}
\subsection{Digitale Technologie: Treiber der digitalen Transformation} %%%%%%%%%%%%%%%%%%%%%%%%%%%%%%%%%%%%%%%%%%%%%%%%%%%%%%%%%
\begin{itemize}
   \item Cloud Computing (Wiederholung von VL3)
   \item[] \includegraphics[scale=0.4]{repeat.png}
   \item[] \hspace*{0.5cm} \includegraphics[scale=0.3]{CloudComputingKonzept.png}
   
   \item \textbf{Internet of Things}:\\
         Erweitertes Internet, in dem neben klassischen Rechnern und mobilen Endgeräten auch beliebige physische Gegenstände eingebunden werden

   \item \textbf{Augmented Reality}:
      \begin{itemize}
			\item Erweiterte Realität\\
			      Computergestützte Erweiterung der Realitätswahrnehmung
			\item Beispiel: mit App und Kamera Möbel virtuell in physischem Zimmer platzieren
      \end{itemize}

   \item \textbf{Blockchain}:
      \begin{itemize}
			\item Elektronisches Register (Liste) von Datensätzen (verteilte, öffentliche Datenbank)
			\item Dezentral verwaltet $\rightarrow$ sicher
			\item Blöcke (neue mit alten) werden unveränderbar miteinander verkettet
      \end{itemize}

   \item \textbf{Kerneigenschaften digitaler Technologien}:
      \begin{itemize}
			\item Homogenität der Daten:\\
			      Verschiedene Dateiformate können für verschiedene Zwecke genutzt werden
			\item Re-Programmierbarkeit:\\
			      Technologie kann für verschiedene Zwecke eingesetzt werden
			\item Selbstreferenzierung:\\
			      z.B. Kindle nutzt die Amazon Cloud zum Speichern der Bücher wodurch abhängige Netzwerkeffekte zu digitalen Innovation entstehen
      \end{itemize}
\end{itemize}


\vspace*{0.5cm}
\subsection{Wertschöpfungsstrukturen verändern} %%%%%%%%%%%%%%%%%%%%%%%%%%%%%%%%%%%%%%%%%%%%%%%%%%%%%%%%%%%%%%%%%%%%%%%%%%%%%%%%%
\begin{itemize}
   \item \textbf{Veränderungen von Wertschöpfung durch digitale Transformation}:
   \item[] \includegraphics[scale=0.40]{veraenderung.png}
   
   \item \textbf{Digitale Fertigkeiten eines Unternehmens}:\\
         Digitale Daten und Informationstechnologien in seine Produkte, Dienstleistungen, Geschäftsprozesse und organisatorischen Systeme zu integrieren und so einen Mehrwert zu generieren
		\begin{itemize}
			\item IT-Unternehmenspartnerschaften
			\item Externe IT-Verbindungen
			\item Strategische Ausrichtung der IT
			\item IT Geschäftsprozessintegration
			\item IT Management
			\item IT Infrastruktur
		\end{itemize}

   \item \textbf{Wechselwirkung Produzent $–$ Konsument}:
   \item[] \includegraphics[scale=0.33]{prozess.png}
   
   \item \textbf{Modulare Architekturen}:\\
         Aufteilung eines Produktes in möglichst unabhängige Module (Verbund über standardisierte Schnittstellen)\\
         $\rightarrow$ Flexibilität
         
   \item \textbf{Consumerization}:
      \begin{itemize}
			\item Definition aus Vorlesung: \\
			Consumerization bezeichnet den spezifischen Einfluss, den verbraucherorientierte Technologien auf Unternehmen haben können. Sie spiegelt wider, wie Unternehmen von neuen Technologien und Modellen, die aus dem Konsumbereich und nicht aus dem Unternehmens-IT-Sektor stammen, beeinflusst werden und diese nutzen können.
			\item Wikipedia: \\
			Consumerization bezeichnet den Prozess bzw. die Erscheinung, dass elektronische Endgeräte, wie beispielsweise Smartphone, Tablet-PCs, von Arbeitnehmern auch für ihre Erwerbsarbeit benutzt werden.
      \end{itemize}
      

   \begin{minipage}[t]{0.4\textwidth}

	\textbf{Vorteile Consumerization} 
	\begin{itemize}
	\item bestimmte Arbeiten lassen sich dezentralisieren und flexibler organisieren und durchführen
	\item mehr Kontrolle der Arbeitnehmer über ihre Zeit und Arbeitsbeziehungen
	\end{itemize}

\end{minipage}\begin{minipage}[t]{0.1\textwidth}
   \ 
\end{minipage}\begin{minipage}[t]{0.4\textwidth}

	\textbf{Nachteile Consumerization} 
	\begin{itemize}
	\item auflösende Grenze zwischen Berufs- und Privatleben
	\item geringere Kontrollmöglichkeiten der Unternehmen
	\item Firmen können über die Netzwerkverbindungen auf die privat genutzten Geräte zugreifen
	\item Sicherheitsprobleme
	\end{itemize}


\end{minipage}
\end{itemize}


\vspace*{0.5cm}
\subsection{Künstliche Intelligenz} %%%%%%%%%%%%%%%%%%%%%%%%%%%%%%%%%%%%%%%%%%%%%%%%%%%%%%%%%%%%%%%%%%%%%%%%%%%%%%%%%%%%%%%%%%%%%
\begin{itemize}
   \item Datenbasierte Verfahren sind auf dem Vormarsch
   \item Im Gegensatz zu \emph{traditionellen} Systementwicklungsprojekten fällt die Auseinandersetzung mit der Unternehmenssituation intensiver aus.
   \item Machine Learning gehört nicht zur schwachen KI. Richtung
   \item 
\end{itemize}

\end{document}