\documentclass[12pt,a4paper]{article}
\usepackage[utf8]{inputenc}
\usepackage[left=2.5cm,right=2.5cm,top=3cm,bottom=2cm]{geometry}
\author{Pauline Speckmann}
\usepackage{graphicx}
\usepackage{booktabs}

\usepackage{fancyhdr}
\pagestyle{fancy}
\fancyhf{}
\fancyhead[l]{Digitalisierung Vorlesung 3 $-$ Zusammenfassung von Pauline Speckmann}
\fancyhead[r]{\thepage}

\begin{document}
\setcounter{section}{2}
\section{Management von Informationssystemen}


\vspace*{1cm}
\subsection{Integrationsorientierte Informationssysteme} %%%%%%%%%%%%%%%%%%%%%%%%%%%%%%%%%%%%%%%%%%%%%%%%%%%%%%%%%%%%%%%%%%%%%%%%
\begin{itemize}
   \item \textbf{Digitaler Zwilling}:\\
         Digitale Darstellung eines realen Objekts oder Systems (materiell oder immateriell)

   \item \textbf{Integrationsansätze}
      \begin{itemize}
			\item \textbf{Datenintegration}:\\
			      Datenbestände von mehreren Informationssystemen werden zentral gespeichert (nicht mehrfach)
			\item \textbf{Funktionsintegration}:\\
			      Mehrere Funktionen werden in einem Informationssystem gebündelt
			\item \textbf{Prozess- oder Vorgangsintegration}:\\
			      In einem Prozess aufeinander folgende Funktionalitäten sind über ein Informationssystem nahtlos miteinander verbunden (Schnittstellen)
		\end{itemize}

   \item \textbf{SAP}: Global führender Anbieter von ERP-Systemen\\
         Beispiel: Verarbeitung eines Kundenauftrags
      \begin{enumerate}
			\item Kundenauftrag wird erfasst
			\item Automatisches Ausführen von: Bestellung der Rohmaterialien, Erzeugung von Fertigungsaufträgen, Übermittlung an die Finanzplanung
			\item Rollen \& Rechte verteilen
      \end{enumerate}

   \item \textbf{ERP- (Enterprise-Ressource-Planning) Systeme}:\\
         Integrierte betriebswirtschaftliche Softwarelösungen, die eine Vielzahl Geschäftsprozesse eines Unternehmens abdecken
      \begin{itemize}
			\item Hohe Datenintegration: Zentrale Datenbank
			\item Hohe Funktions- und Prozessintegration: Schnittstellen
      \end{itemize}
   \item[] \includegraphics[scale=0.35]{digi_pic.png}
\end{itemize}


\vspace{0.5cm}
\subsection{Auswahl von Informationssystemen} %%%%%%%%%%%%%%%%%%%%%%%%%%%%%%%%%%%%%%%%%%%%%%%%%%%%%%%%%%%%%%%%%%%%%%%%%%%%%%%%%%%
\begin{itemize}
   \item \textbf{Systembereitstellung $–$ Goldene Regeln}:
   \item[] \includegraphics[scale=0.52]{GoldeneRegeln.png}
   
   \item \textbf{Softwareindustrie}:
      \begin{itemize}
			\item Direkte und Indirekte Netzeffekte:\\
			      Der Nutzen eines Programms für einen einzelnen Kunden steigt häufig mit der Gesamtzahl der Nutzer.
			\item Keine Vervielfältigungskosten:\\
			      Hohe initiale Entwicklungskosten, anschließend jedoch nahezu kostenfreie Ver\-viel\-fält\-ig\-ungs\-mög\-lich\-keit\-en (Fixkostendegression)
			\item Kein Wertverlust durch Gebrauch
			\item Make or buy?
         \item[] \includegraphics[scale=0.5]{MakeOrBuy.png}
%         \item[] \begin{minipage}[t]{0.43\textwidth} \vspace*{0cm}
%                     \begin{center}
%                        \textbf{Make}:\\
%                        Eigenentwickelte Software
%                     \end{center}
%                     \begin{itemize}
%                        \item Nahezu vollständige Abdeckung unternehmensspezifischer Anforderungen
%                        \item Vollständige Integration in die Gesamtheit bereits implementierter Anwendungen
%                        \item Kosten für Anpassung und Einführung entfallen weitestgehend
%                     \end{itemize}
%                  \end{minipage}
%                  \begin{minipage}[t]{0.45\textwidth} \vspace*{0cm}
%                     \begin{center}
%                        \textbf{Buy}:\\
%                        Fremdentwickelte Software
%                     \end{center}
%                     \begin{itemize}
%                        \item Eliminierung der Entwicklungszeiten durch rasche Produktverfügbarkeit
%                        \item Reduzierung der Einführungs- und Übergangszeit im Vergleich zu Individual-Software
%                        \item Gewährleistung der Weiterentwicklung durch den Anbieter
%                        \item Unabhängigkeit der Programmentwicklung von der Verfügbarkeit der IT-Ressourcen
%                     \end{itemize}
%                  \end{minipage}


\newpage %Manuelle Formatierung
         \item \textbf{Kostenvergleichsrechnung}:
         \item[] \includegraphics[scale=0.4]{Vergleichskosten.png}         
         
         \item \textbf{Nutzenkategorien von Informationssystemen}:
         \item[] \includegraphics[scale=0.45]{Nutzen.png}
      \end{itemize}
   
   \item \textbf{Anwendungslebenszyklus}:\\
      \begin{minipage}[t]{0.3\textwidth}\vspace*{0cm}
         \begin{enumerate}
				\item Entwicklung
				\item Einführung
				\item Wachstum
				\item Sättigung / Reife
				\item Rückgang
				\item Abschaffung
         \end{enumerate}
      \end{minipage}
      \begin{minipage}[t]{0.3\textwidth}\vspace*{0cm}
         \includegraphics[scale=0.5]{Anwendungsleben.png}
      \end{minipage}
      
      
      
\end{itemize}


\vspace{0.5cm}
\subsection{Erstellung von Individualsoftware} %%%%%%%%%%%%%%%%%%%%%%%%%%%%%%%%%%%%%%%%%%%%%%%%%%%%%%%%%%%%%%%%%%%%%%%%%%%%%%%%%%
\begin{itemize}
   \item \textbf{Planung eines Softwareentwicklungsprozesses}:
      \begin{enumerate}
			\item Anforderungsanalyse und Erstellung einer Spezifikation
			\item Design
			\item Entwicklung
			\item Test und Integration
			\item Auslieferung des Produkts
			\item Wartung und Support
      \end{enumerate}
   
   \item \textbf{Strukturgetriebene Softwareentwicklung: Spiralmodell}\\
         Wiederholender Durchlauf von Entwicklungsphasen in Iterationen von jeweils 4 Schritten mit kontinuierlicher Bereitstellung von Prototypen.
      \begin{enumerate}
         \item \textbf{Analyse}:\\
                Definition von Rahmenbedingungen, Zielen, Anforderungen und Lösungsalternativen, Freigabe zur Umsetzung
         \item \textbf{Evaluierung}:\\
                Evaluierung der umgesetzten Lösungsalternativen. Darauf basierend Erkennung von Risiken und Erarbeitung adäquater Strategien zur Vermeidung der Risiken.
         \item \textbf{Realisierung}:\\
                Definition und anschließende Realisierung des Vorgehens, basierend auf den identifizierten Risiken.
         \item \textbf{Planung}:\\
                Review der vorangegangenen Schritte und Planung der nächsten Iteration
      \end{enumerate}
   
   \item \textbf{Prinzipien agiler Softwareentwicklung}:
      \begin{itemize}
			\item Transparenz und Geschwindigkeit der Entwicklung erhöhen\\
			      Reaktion auf Änderungen $>$ Verfolgung eines festgelegten Plans
			\item Fehler minimieren\\
			      Funktionierende Software $>$ Umfangreiche Dokumentation
			\item Kommunikation und Interaktion!\\
			      Kooperation mit Projektbetroffenen $>$ Vertragsverhandlungen\\
			      Individuen und Interaktionen $>$ Prozesse und Tools
      \end{itemize}
   
   \item \textbf{SCRUM}:
      \begin{itemize}
			\item Modell der agilen Softwareentwicklung
			\item Transparenz, Überprüfung und Anpassung
			\item Grober, zeitlicher Rahmen wird definiert und dann angepasst\\
			      $\rightarrow$ Sprint Planning
			\item Teams sind selbstorganisiert\\
			      $\rightarrow$  Scrum Master, Product Owner, Team\\
			      $\rightarrow$ Daily SCRUM Meetings
      \end{itemize}
   
   \item \textbf{DevOps}:
      \begin{itemize}
			\item Development + Operations
			\item \textbf{Ziel}: In sich verändernden Umgebungen mit schlanken und flexiblen Software-Entwicklungsprozessen schnell zu reagieren
			\item DevOpszur Integration von Entwicklung und Betrieb:
			\item[] \includegraphics[scale=0.45]{DevOps.png}
			
			\item \textbf{Limitationen und Herausforderungen von DevOps}:
			   \begin{itemize}
					\item Flexibilität
					\item Automatisierung
					\item Lean-Prinzipien $\rightarrow$ System optimieren
					\item Alignment-Herausforderung $\rightarrow$ Überwachung der wichtigsten Indikatoren
					\item Kultur- und Wissensaustausch
            \end{itemize}
      \end{itemize}
   
   \item \emph{Magisches Dreieck} des Projektmanagements
   \item[] \includegraphics[scale=0.5]{magic.png}
\end{itemize}


\vspace{0.5cm}
\subsection{Beschaffung von Standardsoftware} %%%%%%%%%%%%%%%%%%%%%%%%%%%%%%%%%%%%%%%%%%%%%%%%%%%%%%%%%%%%%%%%%%%%%%%%%%%%%%%%%%%
\begin{itemize}
   \item \textbf{Vorgehen zur Softwareauswahl}:
      \begin{enumerate}
			\item Ist-Analyse
			\item Definition der Anforderung
			\item Marktanalyse
			\item Vergleich der Angebote
			\item Vertragsverhandlung
      \end{enumerate}
      
\newpage %Manuelle Formatierung
   \item \textbf{Kriterien für die Softwareauswahl}:
   \item[] \includegraphics[scale=0.45]{KriterienSoftwareauswahl.png}
   
   \item \textbf{Proprietäre vs. Open Source Software}:
   \item[] \includegraphics[scale=0.48]{foss.png}
   
   \item \textbf{IT-Outsourcing: Vor-und Nachteile}
   \item[] \includegraphics[scale=0.5]{out.png}
   
   \item \textbf{Cloud Computing}:\\
         Dynamische Bereitstellung von IT-Ressourcen über das Internet zur schnelleren Innovation und für flexiblere Ressourcen / Skaleneffekte
      \begin{itemize}
         \item \textbf{Infrastructure-as-a-Service (IaaS)}:\\
               Umfasst alle IT-Leistungen der Basisinfrastruktur z.B.Rechnerkapazitäten, Netzwerke und Speicherplatz.
         \item \textbf{Platform-as-a-Service (PaaS)}:\\
               IT-Leistungen, mit denen sich Anwendungssoftware und -komponenten entwickeln und integrieren lassen.
         \item \textbf{Software-as-a-Service (SaaS)}:\\
               Anwendungen und Dienste, die über Cloud Dienstebereitgestellt werden.
      \end{itemize}
\end{itemize}


\newpage
\subsection{QUIZFRAGEN} %%%%%%%%%%%%%%%%%%%%%%%%%%%%%%%%%%%%%%%%%%%%%%%%%%%%%%%%%%%%%%%%%%%%%%%%%%%%%%%%%%%%%%%%%%%%%%%%%%%%%%%%%%%%%%
\begin{itemize}
   \item ERP-Systeme sind modular aufgebaut.
   
   \item Das Ziel der Prozess- oder Vorgangsintegration ist ursprünglich voneinander isolierte Prozesse aneinander anzugleichen oder auch zu verknüpfen.
   
   \item Vorteile von Standardsoftware (im Vergleich zu eigenentwickelter) sind die Gewährleistung der Programmwartung und -weiterentwicklung durch den Anbieter und der Profit vom \emph{Know-How}, das von vielen Anwendern in der Software abgebildet ist.
   
   \item  Bei Cloud Software werden Nutzungsentgelte verrechnet, aber es entstehen keine Wartungskosten für das nutzende Unternehmen.
   
   \item Die Zusammenarbeit verschiedenster Entwickler birgt ein immenses Innovationspotential bei Open Source Software.
   \item Bei Open Source Software kann der Quellcode von jedermann eingesehen, verändert, manipuliert und ausgebaut werden.
         Dabei gibt es weder Garantien noch einen klassischen Support.
         
   \item Wenn Unternehmen auf ein Höchstmaß an technischer und organisatorischer Integrität bestehen, sollten Sie die Software eigenstaendig entwickeln.
   \item Bei eigenetwickelter Software ist die Integration der Software unkompliziert, da die Software an die Prozesse angepasst wird.
   
   \item Agile Vorgehensmodelle haben eine gute Einsetzbarkeit bei unklaren Zielen und sich ändernden Anforderungen, erhöhten Kommunikations- und Abstimmungsaufwand, hohe Flexibilität und verringerte Komplexität der Projektverwaltung.
   
   \item SCRUM basiert auf der Grundannahme, dass eine detaillierte Planung zu Beginn wenig Sinn ergibt, da Projekte schlichtweg zu komplex sind.
   
   \item Cloud Computing unterscheidet sich vom IT-Outsourcing, indem lediglich einzelne Anwendungen ausgelagert werden, der Kern der IT aber im Unternehmen verbleibt.
\end{itemize}

\end{document}