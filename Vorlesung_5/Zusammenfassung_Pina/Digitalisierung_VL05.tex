\documentclass[a4]{scrartcl}

\usepackage[ngerman]{babel}
\usepackage[utf8]{inputenc}
\usepackage{mathtools}
\usepackage{amsmath}
\usepackage{amssymb}
\usepackage{geometry}
\usepackage{scrpage2}
\pagestyle{scrheadings}
\clearscrheadfoot


\geometry{
  paper=a4paper, % Change to letterpaper for US letter
  top=2cm, % Top margin
  bottom=1.5cm, % Bottom margin
  left=2cm, % Left margin
  right=3cm, % Right margin
  %showframe, % Uncomment to show how the type block is set on the page
}

\setlength{\parindent}{0em}

\ohead{\\
Pina Kolling}

\begin{document}

\section*{Vorlesung 5}

\textbf{Sichten auf Transformationsstratiegien}
\begin{itemize}
\item \textbf{Innovationsperspektive:}

\begin{itemize}
\item \textbf{Phase 1 Experimentieren am Rande der Organisation}

\begin{itemize}
\item ergänzende Experimente: \\
Das bestehende Geschäftsmodell bleibt bestehen, Innovation beruht auf das vorhandene Geschäftsmodell.
\item disruptive Experimente: \\
Das Geschäftsmodell wird grundlegend innoviert. \\

\hspace*{0.8em} \includegraphics[scale=0.25]{experiment.png} \\


\includegraphics[scale=0.22]{bh.png}

\end{itemize}

\item \textbf{Phase 2 Kollision im Kern}

\begin{itemize}
\item Kollision in der Strategie

\begin{itemize}
\item z.B: Kollision mit Strategien von anderen etablierten Unternehmen der Branche
\end{itemize}

\item Kollision der Organisation \\
Effizienz-Nachteile in:

\begin{itemize}
\item Durchlaufzeiten
\item Entscheidungsgeschwindigkeit
\item Führungsmodelle
\end{itemize}



\end{itemize}





\item \textbf{Phase 3 Neuerfindung an der Wurzel}

\begin{itemize}
\item Veränderung der Kernelemente des Geschäftsmodells durch digitale Technologien
\end{itemize}

\end{itemize}

\item \textbf{Architekturperspektive}

\begin{itemize}
\item Geschwindigkeit  $\rightarrow$ hoher Innovationsgrad
\item Stabilität $\rightarrow$ zuverlässige Kernprozesse
\end{itemize}



\item \textbf{Führungsperspesktive}

\includegraphics[scale=0.3]{ceo.png}




\end{itemize}

\textbf{CIO und CDO}

\begin{itemize}
\item CIO entwickelt IT-Strategie (IT-Expertenwissen)
\item CDO (Digital-Strategisches Geschäftswissen)
\end{itemize}

\includegraphics[scale=0.3]{ciocdo.png}

\ \\

\textbf{Entscheidungen einer Transformationsstrategie}

\includegraphics[scale=0.27]{dec.png}

\ \\

\includegraphics[scale=0.4]{8.png}

\ \\

\includegraphics[scale=0.3]{altneu.png}

\ \\

\textbf{Digitale Transformation Leitfragen}

\begin{itemize}
\item Welche Technologien sind von zentraler Bedeutung für das Unternehmen?
\item Mit welchen digitalen Angeboten und Prozessen werden zukünftig Erlöse generiert?
\item Wie wird das Digitalgeschäft aufgebaut und geführt, welche strukturellen Anpassungen sind im Unternehmen erforderlich?
\item Welche Investitionsmittel stehen zur Finanzierung des digitalen Transformationsvorhabens zur Verfügung?
\end{itemize}

\includegraphics[scale=0.3]{ziele.png}






\end{document}